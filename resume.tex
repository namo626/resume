\documentclass[10pt]{article}

\usepackage[hidelinks]{hyperref}
\usepackage{titlesec}
\usepackage{titling}
\usepackage[left=1.0cm, right=1.0cm, top=0.7cm, bottom=1cm]{geometry}
%\usepackage{mathpazo}
%\usepackage[default]{gfsbodoni}
%\usepackage{mlmodern}
%\usepackage[T1]{fontenc}
\usepackage{enumitem}
\usepackage[skins,breakable]{tcolorbox}
\newlist{myitemize}{itemize}{5}
\setlist[myitemize]{label=\textbullet,labelsep=.5cm,leftmargin=3cm,font=\sc,itemsep=0pt,topsep=0pt}
\pagenumbering{gobble}

\titleformat{\section}
{\normalsize\bfseries\uppercase}
{}
{0em}
{}[\titlerule]

\titleformat{\subsection}[runin]
{\bfseries}
{}
{0em}
{}

\titleformat{\subsubsection}
{\bfseries}
{}
{0em}
{}

% \titlespacing{\subsubsection}
% {0pt}{1em}{0.1em}

\renewcommand{\maketitle}{
  \begin{center}
    {\Large\bfseries\theauthor}

    namo@utexas.edu --- (559) 545-2025 \\
    \url{github.com/namo626}
  \end{center}
}
\newenvironment{body}{}
%{\begin{tcolorbox}[blanker,breakable,left=1.5cm]}
%  {\end{tcolorbox}}
%{\hspace{.1\textwidth} \begin{trivlist}\leftskip=.9\textwidth}
%{\end{trivlist}}

\begin{document}

\title{Resume}
\author{CHAYANON (NAMO) WICHITRNITHED}
\maketitle

\section{Education}
\begin{body}
  \textbf{The University of Texas at Austin - Oden Institute for Computational Engineering and Sciences} \hfill 2020 - Present \\
   Ph.D. Computational Science, Engineering, and Mathematics \\
   Advisor: Clint Dawson \\
    GPA: 3.9/4.0
  \begin{itemize}[leftmargin=*]
    \setlength\itemsep{-0.3em}
    \item Research areas: Simulation of hurricane storm surge and coastal floods, numerical methods for fluids
    \item Relevant coursework: High-Performance Computing, Stabilized Methods for CFD, Numerical Analysis, Fluid Dynamics, Conservation Laws, Applied Analysis
  \end{itemize}

\noindent
  \textbf{Georgia Institute of Technology} \hfill August 2016 - May 2020 \\
  B.S. Physics with Highest Honors, with minor in Scientific \& Engineering Computing \\
  GPA: 3.94/4.00
\end{body}
% \section{Research Areas}
% \begin{body}
%   Storm surge simulations, finite element methods, high performance computing
% \end{body}
\section{Experience}
\begin{body}

  \textbf{Graduate Research Assistant} --- Austin, TX \hfill \textmd{May 2021 - Present} \\
  \textit{Computational Hydraulics Group, Oden Institute}
  \begin{itemize}[leftmargin=*]
    \setlength\itemsep{-0.3em}
  \item Implement and test parametric rainfall models on a discontinuous Galerkin variant of the ADvanced CIRCulation (ADCIRC) 2D finite element model to better capture the interaction of various flooding sources in compound floods
  \item Developing a coupled continuous/discontinuous Galerkin finite element method based on ADCIRC to improve simulation of advection-dominated flows while maintaining efficiency
  \item Partner with the Texas Water Development Board to determine worst-case scenario flood levels by running parallel ADCIRC simulations for historical storms with extreme levels of surge or river discharge
  \item Prepared and validated compact finite element meshes of the Gulf of Mexico using OceanMesh2D and QGIS for use in the projects above as well as the collaborative Multiphysics Simulations and Knowledge discovery through AI/ML (MuSiKAL) project
  \end{itemize}
  \textbf{Undergraduate Research Assistant} --- Atlanta, GA \hfill \textmd{2017 - 2020} \\
  \textit{Pattern Formation and Control Laboratory, Georgia Tech}
\begin{itemize}[leftmargin=*]
  \setlength\itemsep{-0.3em}
\item Generated numerical simulations of quasi-2D turbulent flows and their visualizations in MATLAB

\item Performed particle image velocimetry (PIV) to compare experimental data with simulation
\item Optimized and tested regression techniques for estimating physical parameters of noisy quasi-2D flows

\item
Implemented  and tuned recurrent neural networks (RNNs) to predict chaotic trajectories of dynamical systems

\end{itemize}

\section{Technical Skills}
\begin{itemize}[leftmargin=*]
  \setlength\itemsep{-0.3em}
\item \textbf{Programming:}
C,C\texttt{++}, Fortran, MPI, OpenMP, Python, MATLAB, Mathematica, Bash, HTML
\item \textbf{Tools:}
GNU/Linux, Git, GNU Make, CMake, \LaTeX, QGIS, Tracker
\item \textbf{Research:}
ADvanced CIRCulation model (ADCIRC), Figuregen, OceanMesh2D
\end{itemize}

\end{body}
\section{Conference Presentations}
\begin{body}
  \begin{itemize}[leftmargin=*]
    \setlength\itemsep{-0.3em}
   \item \textit{Developing a Compound Flood Model using the Discontinuous Galerkin Method}. Planet Texas 2050 Conference. The University of Texas at Austin, Austin, TX, April 2022.
  \item \textit{The Impact of Boundary Conditions on Spectral Condensation of Turbulence: Numerics and Experiment}.
  $71^{st}$ Annual Meeting of the APS Division of Fluid Dynamics. Georgia World Congress Center, Atlanta, GA, November 2018.
  \end{itemize}

\end{body}
\section{Honors \& Awards}
\begin{itemize}[leftmargin=*]
  \setlength\itemsep{-0.3em}
  \item National Initiative for Modeling and Simulation (NIMS) Graduate Fellowship, 2020 - Present
  \item Runners Up - Planet Texas 2050 Symposium Student Poster Competition, April 2022
  \item Faculty Honors, Fall 2016 - Spring 2018
\end{itemize}

\section{Additional information}
\begin{itemize}[leftmargin=*]
  \setlength\itemsep{-0.3em}
\item \textbf{Languages:} Native proficiency in Thai
\item \textbf{Work Eligibility:} Extended eligibility to work in the U.S.
\end{itemize}
\end{document}
