\documentclass[11pt]{article}

\usepackage[hidelinks]{hyperref}
\usepackage{titlesec}
\usepackage{titling}
\usepackage[left=1.0cm, right=1.0cm, top=1.5cm, bottom=2cm]{geometry}
%\usepackage{mathpazo}
%\usepackage[default]{gfsbodoni}
%\usepackage{mlmodern}
%\usepackage[T1]{fontenc}
\usepackage{enumitem}
\usepackage[skins,breakable]{tcolorbox}
\newlist{myitemize}{itemize}{5}
\setlist[myitemize]{label=\textbullet,labelsep=.5cm,leftmargin=3cm,font=\sc,itemsep=0pt,topsep=0pt}
\pagenumbering{gobble}

\titleformat{\section}
{\normalsize\bfseries\uppercase}
{}
{0em}
{}[\titlerule]

\titleformat{\subsection}[runin]
{\bfseries}
{}
{0em}
{}

\titleformat{\subsubsection}
{\bfseries}
{}
{0em}
{}

% \titlespacing{\subsubsection}
% {0pt}{1em}{0.1em}

\renewcommand{\maketitle}{
  \begin{center}
    {\Large\bfseries\theauthor}

    namo@utexas.edu --- (559) 545-2025 \\
    \url{github.com/namo626}
  \end{center}
}
\newenvironment{body}{}
%{\begin{tcolorbox}[blanker,breakable,left=1.5cm]}
%  {\end{tcolorbox}}
%{\hspace{.1\textwidth} \begin{trivlist}\leftskip=.9\textwidth}
%{\end{trivlist}}

\begin{document}

\title{Resume}
\author{CHAYANON (NAMO) WICHITRNITHED}
\maketitle

\section{Education}
\begin{body}
  \textbf{The University of Texas at Austin} \hfill May 2020 - Present\\
  Oden Institute for Computational Engineering and Sciences \\
  Ph.D. Computational Science, Engineering, and Mathematics \\
  Advisor: Clint Dawson \\
  GPA: 3.9/4.0 \\
  Research areas: storm surge and coastal simulation, numerical methods for fluids, high performance computing \\

\noindent
  \textbf{Georgia Institute of Technology} \hfill August 2016 - May 2020 \\
  B.S. Physics with Highest Honor \\
  Minor in Scientific \& Engineering Computing \\
  GPA: 3.94/4.00
\end{body}
% \section{Research Areas}
% \begin{body}
%   Storm surge simulations, finite element methods, high performance computing
% \end{body}
\section{Research Experience}
\begin{body}

  \textbf{Graduate Research Assistant} \hfill \textmd{May 2021 - Present} \\
  \textit{Computational Hydraulics Group, Oden Institute}
  \begin{itemize}[leftmargin=*]
    \itemsep0em
  \item Developing a coupled continuous/discontinuous Galerkin method based on the ADvanced CIRCulation model (ADCIRC) to improve modeling of storm surge and coastal flooding
  \item Compound flood modeling: implement and test parametric rainfall models on a discontinuous Galerkin variant of ADCIRC to better capture the interaction of various flooding sources
  \item Historic Flood level project: working on a project with the Texas Water Development Board (TWDB) on determining worst-case scenario flood level by running ADCIRC simulations for historical storms. Modified a large FEM mesh to a more compact one for use in this project.
  \item Parallel HEC-RAS: help implementing an MPI-parallel version of the River Analysis System (HEC-RAS)
  \item MusiKal project: modified a global FEM mesh to a smalller one for use in this project.
  \end{itemize}
  \textbf{Undergraduate Research Assistant} \hfill \textmd{2017 - 2020} \\
  \textit{Pattern Formation and Control Laboratory, Georgia Tech}
\begin{itemize}[leftmargin=*]
  \itemsep0em
\item Generated MATLAB simulations of quasi-2D turbulent flows and their visualizations.

\item Performed particle image velocimetry (PIV) to compare experimental data with simulation.

\item
Implemented  and tuned recurrent neural networks (RNNs) to predict chaotic trajectories of dynamical systems.
\item Tested and optimized parameter estimation algorithms for quasi-2D flows using simulation and experimental data.

\end{itemize}

\section{Technical Skills}
\begin{itemize}[leftmargin=*]
  \itemsep0em
\item \textbf{Programming}
C,C\texttt{++}, Fortran, MPI, OpenMP, Python, MATLAB, Mathematica, Bash, HTML
\item \textbf{Tools}
GNU/Linux, Git, GNU Make, CMake, \LaTeX, QGIS, Tracker
\item \textbf{Research}
ADvanced CIRCulation model (ADCIRC), Figuregen, OceanMesh2D
\end{itemize}

\end{body}
\section{Conference Presentations}
\begin{body}
  \begin{itemize}[leftmargin=*]
    \itemsep0em
   \item \textit{Developing a Compound Flood Model using the Discontinuous Galerkin Method}. Planet Texas 2050 Conference. University of Texas at Austin, Austin, TX, April 2022.
  \item \textit{The Impact of Boundary Conditions on Spectral Condensation of Turbulence: Numerics and Experiment}.
  $71^{st}$ Annual Meeting of the APS Division of Fluid Dynamics. Georgia World Congress Center, Atlanta, GA, November 2018.
  \end{itemize}

\end{body}
\section{Honors \& Awards}
\begin{itemize}[leftmargin=*]
  \itemsep0em
  \item National Initiative for Modeling and Simulation (NIMS) graduate fellowship, 2020 - 2024
  \item Runners up - Planet Texas 2050 Symposium Student Poster Competition, April 2022
  \item Faculty Honors, Fall 2016 - Spring 2018
\end{itemize}

\section{Additional information}
\begin{itemize}[leftmargin=*]
  \itemsep0em
\item \textbf{Languages} Fluent in Thai
\item \textbf{Work Eligibility} Extended eligibility to work in the U.S.
\end{itemize}
\end{document}
